%% Generated by Sphinx.
\def\sphinxdocclass{report}
\documentclass[letterpaper,10pt,english]{sphinxmanual}
\ifdefined\pdfpxdimen
   \let\sphinxpxdimen\pdfpxdimen\else\newdimen\sphinxpxdimen
\fi \sphinxpxdimen=.75bp\relax

\PassOptionsToPackage{warn}{textcomp}
\usepackage[utf8]{inputenc}
\ifdefined\DeclareUnicodeCharacter
% support both utf8 and utf8x syntaxes
  \ifdefined\DeclareUnicodeCharacterAsOptional
    \def\sphinxDUC#1{\DeclareUnicodeCharacter{"#1}}
  \else
    \let\sphinxDUC\DeclareUnicodeCharacter
  \fi
  \sphinxDUC{00A0}{\nobreakspace}
  \sphinxDUC{2500}{\sphinxunichar{2500}}
  \sphinxDUC{2502}{\sphinxunichar{2502}}
  \sphinxDUC{2514}{\sphinxunichar{2514}}
  \sphinxDUC{251C}{\sphinxunichar{251C}}
  \sphinxDUC{2572}{\textbackslash}
\fi
\usepackage{cmap}
\usepackage[T1]{fontenc}
\usepackage{amsmath,amssymb,amstext}
\usepackage{babel}



\usepackage{times}
\expandafter\ifx\csname T@LGR\endcsname\relax
\else
% LGR was declared as font encoding
  \substitutefont{LGR}{\rmdefault}{cmr}
  \substitutefont{LGR}{\sfdefault}{cmss}
  \substitutefont{LGR}{\ttdefault}{cmtt}
\fi
\expandafter\ifx\csname T@X2\endcsname\relax
  \expandafter\ifx\csname T@T2A\endcsname\relax
  \else
  % T2A was declared as font encoding
    \substitutefont{T2A}{\rmdefault}{cmr}
    \substitutefont{T2A}{\sfdefault}{cmss}
    \substitutefont{T2A}{\ttdefault}{cmtt}
  \fi
\else
% X2 was declared as font encoding
  \substitutefont{X2}{\rmdefault}{cmr}
  \substitutefont{X2}{\sfdefault}{cmss}
  \substitutefont{X2}{\ttdefault}{cmtt}
\fi


\usepackage[Bjarne]{fncychap}
\usepackage{sphinx}

\fvset{fontsize=\small}
\usepackage{geometry}

% Include hyperref last.
\usepackage{hyperref}
% Fix anchor placement for figures with captions.
\usepackage{hypcap}% it must be loaded after hyperref.
% Set up styles of URL: it should be placed after hyperref.
\urlstyle{same}
\addto\captionsenglish{\renewcommand{\contentsname}{Guide d'utilisation}}

\usepackage{sphinxmessages}
\setcounter{tocdepth}{2}



\title{VP Documentation}
\date{Oct 03, 2019}
\release{0.1}
\author{macordier}
\newcommand{\sphinxlogo}{\vbox{}}
\renewcommand{\releasename}{Release}
\makeindex
\begin{document}

\pagestyle{empty}
\sphinxmaketitle
\pagestyle{plain}
\sphinxtableofcontents
\pagestyle{normal}
\phantomsection\label{\detokenize{index::doc}}



\chapter{Introduction}
\label{\detokenize{index:introduction}}
Visual Planning est un outil développé par STILOG qui permet de paramétrer un planning visuel en fonction des besoins d’une société.


\section{Les membres du projet}
\label{\detokenize{index:les-membres-du-projet}}\begin{itemize}
\item {} 
Michel GRATIA -\textgreater{} Directeur des Services informatiques

\item {} 
Julien GUY -\textgreater{}

\item {} 
Maxime CORDIER -\textgreater{}

\end{itemize}


\subsection{Comment se connecter ?}
\label{\detokenize{guide_visual_planning/guide_utilisation:comment-se-connecter}}\label{\detokenize{guide_visual_planning/guide_utilisation::doc}}

\subsubsection{Installer VPDesk}
\label{\detokenize{guide_visual_planning/guide_utilisation:installer-vpdesk}}
VPDesk est le lanceur de l’application visual planning.

Pour le télécharger rendez vous ici : \sphinxurl{https://www.visual-planning.com/fr/accueil-espace-client/espace-clients-vpdesk}

Téléchargez la version windows et exécutez le fichier.


\subsubsection{Compte}
\label{\detokenize{guide_visual_planning/guide_utilisation:compte}}
Votre identifiant est composé de la façon suivante :

\begin{sphinxVerbatim}[commandchars=\\\{\}]
\PYG{n}{Première} \PYG{n}{lettre} \PYG{n}{du} \PYG{n}{prénom} \PYG{o}{+} \PYG{n}{Nom}
\end{sphinxVerbatim}

Lors de votre première connexion laissez votre mot de passe vide, visual planning vous demandera d’en définir un pour votre compte.

Le mot de passe doit être composé de :

\begin{sphinxVerbatim}[commandchars=\\\{\}]
\PYG{o}{\PYGZhy{}} \PYG{n}{Une} \PYG{n}{lettre}
\PYG{o}{\PYGZhy{}} \PYG{n}{Un} \PYG{n}{chiffre}
\PYG{o}{\PYGZhy{}} \PYG{n}{Au} \PYG{n}{moins} \PYG{l+m+mi}{4} \PYG{n}{caractères}
\end{sphinxVerbatim}

\begin{sphinxadmonition}{note}{Note:}
Ce mot de passe n’a pas besoin d’être modifié.
\end{sphinxadmonition}

\begin{sphinxadmonition}{warning}{Warning:}
Si vous n’avez pas encore de compte visual planning ou que vous ne vous souvenez plus de votre mot de passe contactez : \sphinxhref{mailto:vp.support@etf.fr}{vp.support@etf.fr}
\end{sphinxadmonition}


\subsection{Choisir son planning}
\label{\detokenize{guide_visual_planning/guide_utilisation:choisir-son-planning}}
Lors de votre première connexion visual planning vous demandera de sélectionner un planning. Sélectionnez le planning de votre choix.

\begin{sphinxadmonition}{note}{Note:}\begin{itemize}
\item {} \begin{description}
\item[{ETF\_OFFICIEL\_V3 :}] \leavevmode\begin{itemize}
\item {} 
Planification de ressources (Personnel, Matériel)

\item {} 
Gestion des habilitations

\item {} 
Gestion des formations

\end{itemize}

\end{description}

\item {} \begin{description}
\item[{ETF\_RAPPORT\_V3 :}] \leavevmode\begin{itemize}
\item {} 
Rapport soudure et meulage

\item {} 
Plan de veille et gestion des KNs

\end{itemize}

\end{description}

\end{itemize}
\end{sphinxadmonition}

\begin{sphinxadmonition}{warning}{Warning:}
Les autres planning sont des plannings de test et de développement. Aucune données ne pourra être récupérée de ces plannings
\end{sphinxadmonition}


\subsection{Que faire une fois connecté ?}
\label{\detokenize{guide_visual_planning/guide_utilisation:que-faire-une-fois-connecte}}
Une fois la barre de chargement disparu, vous devez demander à Visual Planning de vous afficher des données.


\subsubsection{Se déplacer dans les différents affichages}
\label{\detokenize{guide_visual_planning/guide_utilisation:se-deplacer-dans-les-differents-affichages}}

\subsubsection{Filtrer les données}
\label{\detokenize{guide_visual_planning/guide_utilisation:filtrer-les-donnees}}

\subsection{Import Automatique}
\label{\detokenize{fonctionnalitees/import_automatique:import-automatique}}\label{\detokenize{fonctionnalitees/import_automatique::doc}}

\subsubsection{Chantier}
\label{\detokenize{fonctionnalitees/import_automatique:chantier}}
Import automatique des chantiers renseignés dans KHEOPS


\subsubsection{Personnel}
\label{\detokenize{fonctionnalitees/import_automatique:personnel}}
Import automatique du personnel renseignés dans KHEOPS


\subsubsection{Matériel}
\label{\detokenize{fonctionnalitees/import_automatique:materiel}}
Import automatique du matériel renseignés dans KHEOPS


\subsection{Habilitation et formation}
\label{\detokenize{fonctionnalitees/habilitation_formation:habilitation-et-formation}}\label{\detokenize{fonctionnalitees/habilitation_formation::doc}}

\subsubsection{Préambule}
\label{\detokenize{fonctionnalitees/habilitation_formation:preambule}}
Sur Visual Planning la formation et l’habilitation sont deux événements distinct.

La formation se planifie sur un planning pour “bloquer” les dates et heures de formations de la même manière qu’une planification de chantier ou d’absence.

A la fin de la formation celle-ci peut-être habilitante : cette habilitation doit être ajoutée manuellement sur la personne.

Les habilitations sont attachées à la personne, en cas de retour d’une ressource ou d’une mutation celle-ci récupère les habilitations déjà renseignées.


\paragraph{Pourquoi renseigner les habilitations sur Visual Planning ?}
\label{\detokenize{fonctionnalitees/habilitation_formation:pourquoi-renseigner-les-habilitations-sur-visual-planning}}\begin{itemize}
\item {} 
Chaque secteur ou agence utilisait leur propre solution de suivi dont certaines nécessitait un maintenance lourde et manuelle.

\item {} 
Une base de données communes pour ETF permet de croiser les données.

\item {} 
Au 1er janvier 2020, l’accueil site ou “SECUFER” doit être opérationnel. La dématérialisation de cet accueil site nécessite de pouvoir controler les habilitations de manière beaucoup plus efficace sur les chantier.

\item {} 
Arriver à la dématérialisation des cartes d’habilitations

\end{itemize}


\paragraph{A voir aussi}
\label{\detokenize{fonctionnalitees/habilitation_formation:a-voir-aussi}}
Comment filtrer


\subsubsection{Habilitation}
\label{\detokenize{fonctionnalitees/habilitation_formation:habilitation}}
Les fonctionnalités de gestions des habilitations et formations se trouvent sur le planning \sphinxcode{\sphinxupquote{ETF\_OFFICIEL\_V3}}


\paragraph{L’affichage}
\label{\detokenize{fonctionnalitees/habilitation_formation:l-affichage}}
Rendez-vous sur l’onglet \sphinxcode{\sphinxupquote{affichage}} puis dans la liste des affichages cherchez : \sphinxcode{\sphinxupquote{\_Gestion habilitation personnel}}

Vous arriverez sur cet affichage :

\noindent\sphinxincludegraphics{{affichage_gestion_habilitation}.png}

Cet affichage est découpé en 5 vues :


\subparagraph{Planning des habilitations du personnel}
\label{\detokenize{fonctionnalitees/habilitation_formation:planning-des-habilitations-du-personnel}}\begin{quote}

\noindent\sphinxincludegraphics{{planning_habilitation}.png}
\begin{description}
\item[{Sur cette vue vous allez retrouver :}] \leavevmode\begin{itemize}
\item {} \begin{description}
\item[{Les informations de votre personnels :}] \leavevmode\begin{itemize}
\item {} 
Photo

\item {} 
Matricule (Eurovia)

\item {} 
Nom - Prénom

\item {} 
Qualification de bulletin

\end{itemize}

\end{description}

\item {} 
\sphinxstylestrong{Double cliquer} sur une personne ouvre sa fiche :
\begin{quote}

\noindent\sphinxincludegraphics{{statuts_habilitation}.png}
\end{quote}

\item {} 
\sphinxstylestrong{Double cliquer} sur un événement ouvre sa fiche :
\begin{quote}

\noindent\sphinxincludegraphics{{statuts_habilitation}.png}
\end{quote}

\item {} 
Les couleurs sur les événements correspondent aux statuts

\end{itemize}

\end{description}
\end{quote}


\subparagraph{Liste des événements habilitations}
\label{\detokenize{fonctionnalitees/habilitation_formation:liste-des-evenements-habilitations}}\begin{quote}

\noindent\sphinxincludegraphics{{evenement_habilitation}.png}
\begin{description}
\item[{Par défaut cet affichage est trié avec le filtre}] \leavevmode{[}\sphinxcode{\sphinxupquote{\_Habilitations qui vont être perdues}}{]}\begin{itemize}
\item {} 
Date de fin \textless{} 3 mois

\end{itemize}

\end{description}

\begin{sphinxadmonition}{note}{Note:}\begin{description}
\item[{Vous pouvez aussi utiliser les filtres :}] \leavevmode
\sphinxcode{\sphinxupquote{\_Habilitations suivies}} : toutes les habilitations sauf \sphinxcode{\sphinxupquote{Renouvelée}}, \sphinxcode{\sphinxupquote{Non suivi}} et \sphinxcode{\sphinxupquote{Doublon}}
\sphinxcode{\sphinxupquote{\_Historique des habilitations}} : Toutes les habilitations sans distinction

\end{description}
\end{sphinxadmonition}
\end{quote}


\subparagraph{Liste des statuts}
\label{\detokenize{fonctionnalitees/habilitation_formation:liste-des-statuts}}\begin{quote}

\noindent\sphinxincludegraphics{{statuts_habilitation}.png}
\end{quote}


\begin{savenotes}\sphinxattablestart
\centering
\begin{tabular}[t]{|*{2}{\X{1}{2}|}}
\hline
\sphinxstyletheadfamily 
Libellé
&\sphinxstyletheadfamily 
Condition
\\
\hline\sphinxstartmulticolumn{2}%
\begin{varwidth}[t]{\sphinxcolwidth{2}{2}}
\sphinxstylestrong{Automatique}
\par
\vskip-\baselineskip\vbox{\hbox{\strut}}\end{varwidth}%
\sphinxstopmulticolumn
\\
\hline
+ 6 mois
&
\begin{DUlineblock}{0em}
\item[] Date de fin \textgreater{} 6 mois
\end{DUlineblock}
\\
\hline
Entre 3 et 6 mois
&
\begin{DUlineblock}{0em}
\item[] mois \textless{} Date de fin \textless{} 6 mois
\end{DUlineblock}
\\
\hline
- 3 mois
&
\begin{DUlineblock}{0em}
\item[] Date de fin \textless{} 3 mois
\end{DUlineblock}
\\
\hline
Expirée
&
\begin{DUlineblock}{0em}
\item[] Date de fin \textless{} 0 jour
\end{DUlineblock}
\\
\hline
Doublon
&
\begin{DUlineblock}{0em}
\item[] Date de fin 1 = Date de fin 2
\item[] Habilitation 1 = Habilitation 2
\end{DUlineblock}
\\
\hline
Renouvelée
&
\begin{DUlineblock}{0em}
\item[] Habilitation 1 = Habilitation 2
\item[] L’une des deux est plus récente
\end{DUlineblock}
\\
\hline\sphinxstartmulticolumn{2}%
\begin{varwidth}[t]{\sphinxcolwidth{2}{2}}
\sphinxstylestrong{Manuel}
\par
\vskip-\baselineskip\vbox{\hbox{\strut}}\end{varwidth}%
\sphinxstopmulticolumn
\\
\hline
Suspendu
&
\begin{DUlineblock}{0em}
\item[] Pour suspendre une habilitation
\end{DUlineblock}
\\
\hline
\end{tabular}
\par
\sphinxattableend\end{savenotes}

\begin{sphinxadmonition}{note}{Note:}\begin{itemize}
\item {} 
Le seul statut que vous pouvez positionner manuellement est le statut \sphinxcode{\sphinxupquote{Suspendu}} qui permet  de notifier que la personne est suspendu sur cet habilitaiton.

\item {} 
Les autres statuts sont automatique

\end{itemize}
\end{sphinxadmonition}


\subparagraph{Liste des habilitations}
\label{\detokenize{fonctionnalitees/habilitation_formation:liste-des-habilitations}}\begin{quote}

\noindent\sphinxincludegraphics{{statuts_habilitation}.png}
\end{quote}

Utilisation du glisser/déposer sur le planning habilitation pour ajouter une habilitation à une personne.

\begin{sphinxadmonition}{warning}{Warning:}
Lorsque vous placer une habilitation sur une personne créée manuellement dans Visual Planning, vous aurez un message
d’alerte vous indiquant que celle-ci n’apparaîtra pas dans VBADGE : Aller voir la section Vbadge
\end{sphinxadmonition}
\begin{itemize}
\item {} \begin{description}
\item[{Par défaut vous voyez dans cette vue :}] \leavevmode
\begin{DUlineblock}{0em}
\item[] \sphinxcode{\sphinxupquote{{}`Libellé de l'habilitation}}
\item[] \sphinxcode{\sphinxupquote{Libellé complet de l'habilitation}}, \sphinxcode{\sphinxupquote{durée automatique lors de la planification}}, \sphinxcode{\sphinxupquote{0 = sans recyclage \textbar{} 1 = recyclage}}
\end{DUlineblock}

\end{description}

\item {} 
\sphinxstylestrong{Double cliquer} sur une habilitation pour avoir plus d’informations sur celle-ci :
\begin{quote}

\noindent\sphinxincludegraphics{{fonctionnalitees/../_static/fonctionnalitees/habilitation_formation/editeur_habilitation}.png}
\end{quote}

\end{itemize}


\subparagraph{Tableau de suivi des habilitations}
\label{\detokenize{fonctionnalitees/habilitation_formation:tableau-de-suivi-des-habilitations}}\begin{quote}

\noindent\sphinxincludegraphics{{statuts_habilitation}.png}
\begin{itemize}
\item {} 
Permet d’avoir une vue global des habilitations de son périmètre

\item {} 
Est affiché le nombre de jour restant avant recyclage

\item {} 
Les couleurs correspondent aux statuts des autres vues

\end{itemize}

\begin{sphinxadmonition}{note}{Note:}
\begin{DUlineblock}{0em}
\item[] Utiliser les filtres pour épurer le tableau.
\item[] Exemple 1 : \sphinxcode{\sphinxupquote{Recyclage : Non}} si vous ne voulez pas voir les habilitations ne nécessitant pas de recyclage
\item[] Exemple 2 : \sphinxcode{\sphinxupquote{\_Choix de l'habilitation}} permet de choisir uniquement une habilitation
\end{DUlineblock}
\end{sphinxadmonition}
\end{quote}


\subsection{Vbadge}
\label{\detokenize{fonctionnalitees/vbadge:vbadge}}\label{\detokenize{fonctionnalitees/vbadge::doc}}

\subsubsection{Préambule}
\label{\detokenize{fonctionnalitees/vbadge:preambule}}
Vbadge est une carte distribuée aux collaborateurs permettant grâce à un \sphinxstylestrong{QR code} imprimé dessus de récupérer ses habilitations / autorisations.

Les habilitations / autorisations sont reliées directement à Visual Planning.


\subsubsection{Qui peut avoir une carte ?}
\label{\detokenize{fonctionnalitees/vbadge:qui-peut-avoir-une-carte}}
Seul les personnes importés par KHEOPS peuvent avoir une carte.
\begin{description}
\item[{C’est à dire :}] \leavevmode\begin{itemize}
\item {} 
CDI

\item {} 
CDD

\item {} 
CDIC

\end{itemize}

\end{description}

\begin{sphinxadmonition}{warning}{Warning:}
Les \sphinxcode{\sphinxupquote{Intérimaires}} et \sphinxcode{\sphinxupquote{Stagiaires}} sont les seuls ressources qui ne peuvent pas obtenir de carte.

Vous ne pouvez pas demander une carte pour une personne créée manuellement dans Visual Planning.
\end{sphinxadmonition}

\begin{sphinxadmonition}{note}{Note:}
Pour toute personne créée manuellement et qui est en doublon parce qu’elle a été importée de KHEOPS, vous devez effectuer un {\hyperref[\detokenize{fonctionnalitees/vbadge:rapprochement}]{\sphinxcrossref{\DUrole{std,std-ref}{Rapprochement}}}}
\end{sphinxadmonition}


\subsubsection{Demander une carte pour un collaborateur}
\label{\detokenize{fonctionnalitees/vbadge:demander-une-carte-pour-un-collaborateur}}\begin{description}
\item[{Pour demander une carte vbadge, il faut commencer par vérifier les points suivants concernant le collaborateurs :}] \leavevmode\begin{itemize}
\item {} 
Photo

\item {} 
Nom - Prénom

\item {} 
Qualification de bulletin

\end{itemize}

\end{description}

Ces informations doivent être corrigées avant la demande car elles sont inscrite sur la carte qui sera donnée aux collaborateurs.

Après avoir vérifier les données, vous pouvez faire le demande à …. par mail qui vous envera la carte.


\subsubsection{Rapprochement}
\label{\detokenize{fonctionnalitees/vbadge:rapprochement}}\label{\detokenize{fonctionnalitees/vbadge:id1}}
Dans certains cas, vous avez besoin de planifier la personne avant qu’elle ne soit créée dans KHEOPS.
Pour la plannifier, vous allez la créer manuellement dans Visual Planning mais cela vous bloquera pour faire une demande de carte vbadge.
Lorsque celle-ci sera importé depuis KHEOPS, vous pourrez les \sphinxcode{\sphinxupquote{Rapprocher}} pour fusionner les deux personnes.

\begin{sphinxadmonition}{note}{Note:}
Lors d’un \sphinxcode{\sphinxupquote{Rapprochement}} Tous les événements de la personne créée manuellement sont reporter sur la personne importée de KHEOPS.
\end{sphinxadmonition}

Pour se faire, rendez-vous dans l’onglet \sphinxcode{\sphinxupquote{affichage}} et selectionnez l’affichage : \sphinxcode{\sphinxupquote{\_Gestion des ressources manuelles}}


\paragraph{L’affichage}
\label{\detokenize{fonctionnalitees/vbadge:l-affichage}}\begin{description}
\item[{Il est découpé en 3 vues :}] \leavevmode\begin{itemize}
\item {} 
Personnels créés manuellement

\item {} 
Positionnement du rapprochement

\item {} 
Personnel importés de KHEOPS

\end{itemize}

\item[{Il faut essayer de vider la vue \sphinxcode{\sphinxupquote{Personnels créés manuellement}} en effectuant ces actions :}] \leavevmode\begin{itemize}
\item {} 
Faire un rapprochement entre la personne créée manuellement et la personne importés de KHEOPS

\item {} 
Marquer la ressource comme générique

\item {} 
Spécifier que la personne est un intérimaire

\end{itemize}

\end{description}


\paragraph{Placer un rapprochement}
\label{\detokenize{fonctionnalitees/vbadge:placer-un-rapprochement}}
Prennez la personne créée manuellement à gauche et placer la sur le planning. \sphinxstyleemphasis{Peut importe la date}

Une page s’ouvre dans laquelle vous devez renseigner l’ID VINCI de la même personne importée de KHEOPS.

\begin{sphinxadmonition}{note}{Note:}
Vous retrouvez l’ID VINCI de la personne dans la vue de droite
\end{sphinxadmonition}

Quelques minutes plus tard l’événement aura disparu ainsi que la personne créée manuellement.


\paragraph{Ressource générique}
\label{\detokenize{fonctionnalitees/vbadge:ressource-generique}}
Lorsque vous ouvrez les informations de la personne : \sphinxstylestrong{Double cliquez sur la personne}

Vous trouverez dans la fiche de la personne une case à cocher pour la marquer en \sphinxcode{\sphinxupquote{générique}}.

\begin{sphinxadmonition}{note}{Note:}
Une ressource générique est une ressource non nominative comme : Intérimaire, Poseur de voies, …
\end{sphinxadmonition}


\paragraph{Intérimaire}
\label{\detokenize{fonctionnalitees/vbadge:interimaire}}
Lorsque vous ouvrez les informations de la personne : \sphinxstylestrong{Double cliquez sur la personne}

Vous trouverez dans la fiche de la personne une liste déroulante de type de contrat : Sélectionnez le type \sphinxcode{\sphinxupquote{Intérimaire}}


\subsection{Mise à jour du 02 octobre 2019}
\label{\detokenize{maj/maj_30092019:mise-a-jour-du-02-octobre-2019}}\label{\detokenize{maj/maj_30092019::doc}}

\subsubsection{Suivi des habilitations}
\label{\detokenize{maj/maj_30092019:suivi-des-habilitations}}

\paragraph{Modification des informations de la personne}
\label{\detokenize{maj/maj_30092019:modification-des-informations-de-la-personne}}\begin{itemize}
\item {} \begin{description}
\item[{Ajout d’une rubrique carte pro btp dans la personne}] \leavevmode\begin{itemize}
\item {} 
Permet d’ajouter le scan de la carte pro dans la personne

\item {} 
Permet de récupérer cette carte pro sur Vbadge

\end{itemize}

\end{description}

\item {} \begin{description}
\item[{Ajout d’une case à cocher “Ressource générique” dans la personne}] \leavevmode\begin{itemize}
\item {} 
Permet de catégoriser la ressource en ressource générique*

\end{itemize}

\end{description}

\end{itemize}


\paragraph{Editeur de saisie Personnel}
\label{\detokenize{maj/maj_30092019:editeur-de-saisie-personnel}}\begin{description}
\item[{Modification de l’éditeur de saisie personnel pour les adaptations suivantes :}] \leavevmode\begin{itemize}
\item {} 
Séparation des habilitations valides médicales des autres habilitations

\end{itemize}

\end{description}


\paragraph{Ajout de statut}
\label{\detokenize{maj/maj_30092019:ajout-de-statut}}\begin{itemize}
\item {} 
Ajout d’une catégorie de statuts “Habilitations”

\item {} \begin{description}
\item[{Ajout des ressources suivantes :}] \leavevmode\begin{itemize}
\item {} 
+ 6 mois

\item {} 
Entre 3 et 6 mois

\item {} 
- 3 mois

\item {} 
Expirée

\item {} 
Suspendu

\item {} 
Renouvelée

\end{itemize}

\end{description}

\end{itemize}


\begin{savenotes}\sphinxattablestart
\centering
\begin{tabulary}{\linewidth}[t]{|T|T|T|}
\hline
\sphinxstyletheadfamily 
Libellé
&\sphinxstyletheadfamily 
Condition
&\sphinxstyletheadfamily 
Fonctionnement
\\
\hline
+ 6 mois
&
Date de fin \textgreater{}= 6 mois
&
Export/Import : Auto habilitation \textgreater{} 6 mois
\\
\hline
Entre 3 et 6 mois
&
3 \textless{} Date de fin \textless{} 6
&
Export/Import : Auto habilitation : Entre 3 et 6 mois
\\
\hline
- 3 mois
&
Date de fin \textless{} 3 mois
&
Export/Import : Auto habilitation \textless{} 3 mois
\\
\hline
Expirée
&
Date de fin \textless{} 0
&
Export/Import : Auto habilitation : Expirée
\\
\hline
Renouvelée
&
A un événement d’habilitation plus récent
&
API : Habilitations
\\
\hline
Suspendu
&
Manuellement par l’utilisateur
&
Manuel
\\
\hline
Sans recyclage
&
Habilitation sans recyclage
&
Création de l’événement
\\
\hline
Doublon
&
Habilitations identiques aux même dates
&
API : Habilitation
\\
\hline
\end{tabulary}
\par
\sphinxattableend\end{savenotes}


\subsubsection{Nouveaux affichages}
\label{\detokenize{maj/maj_30092019:nouveaux-affichages}}\begin{itemize}
\item {} 
Séparation de la gestion des habilitations du personnels de la gestion des agréments machine.

\item {} 
Regroupement de la gestion des habilitations et du tableau de suivi.

\end{itemize}


\paragraph{Gestion des habilitations}
\label{\detokenize{maj/maj_30092019:gestion-des-habilitations}}
L’affichage par mois des habilitations n’est pas très lisible et ne permet pas de se rendre compte des habilitations valides de la personne.

L’affichage est maintenant annuel et reprend les couleurs de statut de l’habilitations :

\noindent\sphinxincludegraphics{{affichage_habilitations}.png}


\subsubsection{Modification techniques}
\label{\detokenize{maj/maj_30092019:modification-techniques}}

\paragraph{Hierarchie habilitations}
\label{\detokenize{maj/maj_30092019:hierarchie-habilitations}}\begin{itemize}
\item {} \begin{description}
\item[{Changement du nom de la hierarchie}] \leavevmode\begin{itemize}
\item {} 
Anciennement “Gestion des habilitations PEV” -\textgreater{} “Ajout d’une habilitation / Agrément

\end{itemize}

\end{description}

\item {} \begin{description}
\item[{Ajout du statut}] \leavevmode\begin{itemize}
\item {} 
Uniquement statut de type habilitation

\end{itemize}

\end{description}

\end{itemize}


\paragraph{Automatisation}
\label{\detokenize{maj/maj_30092019:automatisation}}\begin{description}
\item[{Par import/Export}] \leavevmode{[}\sphinxstyleemphasis{(Effectué à 06:00 et 18:00)}{]}\begin{itemize}
\item {} 
Auto habilitation \textgreater{} 6 mois

\item {} 
Auto habilitation : Entre 3 et 6 mois

\item {} 
Auto habilitation \textless{} 3 mois

\item {} 
Auto habilitation : Expirée

\end{itemize}

\end{description}

\begin{sphinxadmonition}{note}{Note:}\begin{description}
\item[{Ces exports ont eu base commune de filtre :}] \leavevmode\begin{itemize}
\item {} 
Uniquement les évenement contenant une ressource \sphinxstylestrong{HABILITATION}

\item {} 
Ne modifie pas les habilitations notées : Ne plus suivre

\item {} 
Ne modifie pas les habilitations : “Suspendu” ou “Renouvelée”

\end{itemize}

\end{description}
\end{sphinxadmonition}
\begin{description}
\item[{Par API}] \leavevmode{[}\sphinxstyleemphasis{(Effectué à 06:00 et 18:00)}{]}\begin{itemize}
\item {} 
Vient vérifier si des recyclages ont été effectués pour passer les habilitations en “Renouvelée”

\item {} 
Vient vérifier si l’habilitation n’est pas un doublon

\end{itemize}

\end{description}


\paragraph{Dimensions de paramétrage}
\label{\detokenize{maj/maj_30092019:dimensions-de-parametrage}}
Les dimensions \sphinxstylestrong{STATUTS} et \sphinxstylestrong{HABILITATIONS} sont a passer en paramétrage pour récupérer les ajouts ou modifications sur les ressources


\paragraph{Modification des habilitations sans recyclage}
\label{\detokenize{maj/maj_30092019:modification-des-habilitations-sans-recyclage}}
Les habilitations sans recyclage passent automatiquement a 20ans de validités pour permettre de les suivres sur le nouveau planning.


\subsubsection{Filtres supplémentaire}
\label{\detokenize{maj/maj_30092019:filtres-supplementaire}}

\paragraph{Filtre d’événement}
\label{\detokenize{maj/maj_30092019:filtre-d-evenement}}

\paragraph{Filtre de Ressources}
\label{\detokenize{maj/maj_30092019:filtre-de-ressources}}
\_Type = Habilitation :


\subsubsection{Modification des fonctionnalités inutilisées}
\label{\detokenize{maj/maj_30092019:modification-des-fonctionnalites-inutilisees}}
Certains événements superflux sont resté dans le planning suite à des mises à jours ou des suppressions de fonctionnalités.


\paragraph{PDC}
\label{\detokenize{maj/maj_30092019:pdc}}\begin{description}
\item[{Suppressions des événements qui permettaient de calculer le plan de charge lors de la plannification horaire :}] \leavevmode\begin{itemize}
\item {} 
Nombre d’événement supprimé : \sphinxstylestrong{26478}

\end{itemize}

\end{description}
\begin{itemize}
\item {} 
La dimension PDC est supprimée

\item {} 
Les hierarchies comprennant encore les ressources PDC ont été modifiées

\item {} 
Les imports/export auto du PDC sont supprimé

\end{itemize}


\paragraph{SOUDURE}
\label{\detokenize{maj/maj_30092019:soudure}}
La soudure à été migrée sur le planning de rapport
\begin{itemize}
\item {} 
Suppression de la dimension soudure

\item {} \begin{description}
\item[{Suppresion des formulaires de soudure :}] \leavevmode\begin{itemize}
\item {} 
Soudure

\item {} 
Meulage

\end{itemize}

\end{description}

\item {} \begin{description}
\item[{Suppression des hierarchies liés à la soudure :}] \leavevmode\begin{itemize}
\item {} 
Rapport soudure

\end{itemize}

\end{description}

\item {} 
Suppression des formulaires VPGO

\end{itemize}


\subsubsection{Import automatique}
\label{\detokenize{maj/maj_30092019:import-automatique}}
Désactivation temporaire de la modification automatique de la qualification de bulletin du personnel.



\renewcommand{\indexname}{Index}
\printindex
\end{document}